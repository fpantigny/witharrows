

%\def\pgfsysdriver{pgfsys-dvips.def}

\input witharrows

\usetikzlibrary{calc}

\let\tiny\relax

Essai de texte.


\vskip1cm

\WithArrowsOptions{up-and-down/width=3cm}

$\WithArrows
A & = B 
\Arrow[up]{an arrow} \\
  & = C + C + C + C + C + C + C + C  \\
  & = C + C + C + C + C + C + C + C
\Arrow[down]{an arrow}\\
  & = E + E 
\endWithArrows$






$\WithArrows[code-after = { \MultiArrow{3,...,5}{essai} }]
A & = B \\
  & = C \\
  & = C \\
  & = E \\
  & = F \\
  & = G \\
\endWithArrows$



\WithArrowsOptions{up-and-down/width=min}

$\WithArrows
A & = B 
\Arrow[up]{essai} \\
  & = C + C + C + C + C + C + C + C  \\
  & = C + C + C + C + C + C + C + C  
 \Arrow[down]{essai} \\
  & = E + E 
\endWithArrows$



\def\essai{avant}

\tikzpicture
\def\essai{modified}
\draw (0,0) -- (1,0) -- (1,1) -- (0,1) -- cycle ;
\endtikzpicture

\essai


\bigskip

\WithArrowsOptions{replace-left-brace-by = [ }


\DispWithArrows[left-brace]
A & = B  \Arrow{essai} \\
  & = C
\endDispWithArrows

\def\frac#1#2{{#1 \over #2}}

\DispWithArrows[displaystyle,wrap-lines,fleqn,mathindent=3cm]
S_n
& = \frac1n \Re \left(\sum_{k=0}^{n-1}\bigl(e^{i\frac{\pi}{2n}}\bigr)^k\right)
\Arrow{sum of terms of a geometric progression of ratio $e^{i\frac{2\pi}n}$}\\
& = \frac1n \Re \left(
\frac{1-\bigl(e^{i\frac{\pi}{2n}}\bigr)^n}{1-e^{i\frac{\pi}{2n}}} \right) 
\Arrow{This line has been wrapped automatically.} \\
& = \frac1n \Re \left(\frac{1-i}{1-e^{i\frac{\pi}{2n}}}\right) 
\endDispWithArrows


$$\WithArrows[ll,interline=5mm,xoffset=5mm,
      tikz-code  = {\draw[rounded corners,
                          every node/.style = {circle,
                                               draw,
                                               auto = false,
                                               inner sep = 1pt,
                                               fill = gray!50,
                                               font = \tiny}] 
                          let \p1 = (#1),
                              \p2 = (#2)
                          in \ifdim \x1 > \x2
                               (\p1) -- node {#3} (\x1,\y2) -- (\p2)
                             \else
                               (\p1) -- (\x2,\y1) -- node {#3} (\p2)
                             \fi ;}]
E & \Longleftrightarrow \frac{(x+4)}3 + \frac{5x+3}5 = 7 
\Arrow{$\times 15$}\\
  & \Longleftrightarrow 5(x+4) + 3(5x+3) = 105 \\
  & \Longleftrightarrow 5x+20 + 15x+9 = 105 \\
  & \Longleftrightarrow 20x+29 = 105 
\Arrow{$-29$}\\
  & \Longleftrightarrow 20x = 76 
\Arrow{$\div 20$}\\
  & \Longleftrightarrow x = \frac{38}{10} 
\endWithArrows$$



\bye


